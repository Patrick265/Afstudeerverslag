\chapter{Inleiding}
Sensor Maritime is gespecialiseerd in het ontwikkelen van sensoren. De specifieke focus ligt op het ontwikkelen van sensorsystemen voor binnenvaartschepen. Sensor Maritime heeft een nieuw product ontwikkeld genaamd Satellite. Satellite is ontwikkeld om een universeel data-driven optimalisatie te maken voor kapiteins van binnenvaartschepen. Satellite is uniek omdat de kapitein plug and play verschillende sensoren kan toevoegen specifiek voor zijn binnenvaartschip. Het geeft mogelijkheden aan de kapitein om zijn transport te verbeteren aan data. Sensor Maritime heeft op dit moment alleen nog maar de hardware ontwikkeld. De vraag aan de afstudeerder is om een generieke applicatie te ontwikkelen voor de hardware om verschillende sensoren te ondersteunen.\newline

\noindent Het doel van dit verslag is om de volgende onderzoeksvraag te beantwoorden \textbf{Hoe kan een modulair softwaresysteem worden ontwikkeld voor de Satellite hardware zo dat verschillende sensoren ondersteund worden en data verstuurd kan worden naar één hoofdsysteem?} Deze vraag wordt beantwoord door verschillende ontwerpen en een testplan te ontwikkelen. De ontwerpen zijn gefocust op een generieke applicatie ontwerp, communicatie. Het testplan is gefocust op hardware validatie. Het product wordt is, geïmplementeerd, getest en uiteindelijk volgt er een demo. \newline

\noindent Om de onderzoeksvraag van dit beantwoorden beschrijft de probleemanalyse de probleemomschrijving, hoofdvraag, deelvragen, doelstellingen en eisen beschreven worden. Na de probleemanalyse komt de aanpak. De aanpak geeft informatie over hoe het project voltooid is. Na de aanpak komt de methode technieken, dit geeft informatie wat er nodig is en wat er gebruikt is om een succesvolle applicatie te maken. Vervolgens wordt er gekeken naar het ontwerp van de applicatie. Daarnaast wordt er gekeken naar het testen van de hardware. Het resultaat zal terugkoppelen op de doelstellingen en eisen. Als laatste geef ik een conclusie en aanbevelingen in het licht van hoofdvraag en deelvragen.