\chapter{Inleiding}
Sensor Maritime heeft een nieuw product ontwikkeld genaamd Satellite. Satellite is ontwikkeld om een universeel data-driven optimalisatie te maken voor kapiteins van binnenvaartschepen. Satellite is uniek omdat de kapitein plug and play verschillende sensoren kan toevoegen specifiek voor zijn binnenvaartschip. Sensor Maritime heeft op dit moment alleen nog maar de hardware ontwikkeld. De vraag aan de stagiair is om een generieke applicatie te ontwikkelen voor de hardware om verschillende sensoren te ondersteunen.\newline

\noindent Het project Satellite heeft het volgende hoofdvraag \textbf{Hoe kan een modulair softwaresysteem worden ontwikkeld voor de Satellite hardware zo dat verschillende sensoren ondersteunt worden en data verstuurd kan worden naar een extern opslagpunt?} Om dit hoofdvraag te beantwoorden zal er worden onderzocht naar generieke applicatie ontwerp, stabiele communicatie en hardware validatie. Het product wordt onderzocht, implementeert, getest en uiteindelijk wordt er een demo gegeven. \newline


\noindent De volgende hoofdstukken komen aan te pas in het verslag. Ten eerste zal er een samenvatting gegeven van het eindverslag. Vervolgens is er een voorwoord toegevoegd, dit geeft een persoonlijk bericht van mijn eigen ervaringen. Daarnaast wordt in het voorwoord ook een dankwoord gegeven. Na het voorwoord wordt er een inleiding gegeven. Vervolgens wordt er basisinformatie gegeven over Sensor Maritime, hierin wordt beschreven wie Sensor Maritime is en wat ze doen. De probleemanalyse is het volgende hoofdstuk, hierin wordt de projectomschrijving, hoofdvraag, deelvragen, doelstellingen en eisen beschreven worden. Na de probleemanalyse komt de aanpak. De aanpak geeft informatie over hoe het project voltooid wordt. Na de aanpak komt de methode technieken, dit geeft informatie wat er nodig is en wat er gebruikt is om een succesvolle applicatie te maken. Vervolgens wordt er gekeken naar het ontwerp van de applicatie. Daarnaast wordt er gekeken naar het testen van de applicatie. Het resultaat zal terugkoppelen op de deelvragen. Als laatste zal er een conclusie en aanbevelingen gegeven worden. Hier wordt teruggekoppeld op de hoofdvraag en deelvragen.