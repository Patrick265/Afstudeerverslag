\chapter{Conclusie en aanbevelingen}
\section{Conclusie}
In dit eindverslag is een onderzoek gedaan naar het volgende onderzoeksvraag: \textbf{hoe kan een modulair softwaresysteem worden ontwikkeld voor de Satellite hardware zo dat verschillende sensoren ondersteund worden en data verstuurd kan worden naar één hoofdsysteem?}. Omdat de onderzoeksvraag uit twee onderdelen bestaat, namelijk generiek applicatie ontwerp en communicatie met één hoofdsysteem. Zal de bevindingen per onderdeel nader uitgelegd worden. Vervolgens zal er ook gekeken worden naar een doelstelling wat niet in het onderzoeksvraag staat, maar wat wel een belangrijk onderdeel is van het uiteindelijke product. \newline

\noindent Uit de resultaten van het onderzoek naar een generieke applicatie ontwerp is naar boven gekomen dat het project het best opgezet kan worden met een lagen structuur, en elke laag heeft een specifieke taak. Met de lagen structuur kunnen er makkelijk aanpassingen gemaakt worden op specifieke punten zonder dat de applicatie logica aangepast moet worden. Het doel is dat applicatie logica zo min mogelijk aangepast moet worden, dit helpt met robuustheid. Naast het lagen structuur is er ook gebruik gemaakt van de preprocessor, de preprocessor is een onderdeel van de C taal wat veel gebruikt is in het ontwerpen en ontwikkelen van een generieke applicatie ontwerp. De preprocessor is op het moment zo opgezet dat makkelijk sensoren uit en aan gezet kunnen worden. Dit wordt gedaan door definities uit of aan te zetten. Met de preprocessor kan dan makkelijk sensoren uit of aangezet worden, maar ook compleet stukken logica. \newline

\noindent Het volgende resultaat van het onderzoek naar een robuuste communicatie. Er is hiervoor gefocust op de CAN communicatie. Met de Satellite is er een probleem ontstaan met het bestaande CAN protocol van Sensor Maritime. De huidige protocol ondersteunde niet meerdere sensor data in één bericht. Om dit te ondersteunen is de payload aangepast van het protocol. Hiervoor zijn drie ontwerpen gemaakt en vergeleken. Uiteindelijk is een ontwerp geïmplementeerd en zal ook gebruikt worden voor toekomstige producten. \newline

\noindent Het laatste onderdeel, is een doelstelling wat Sensor Maritime graag wilt zien. Deze doelstelling is manier ontwikkelen om nieuw Satellite hardware te kunnen testen. Als Sensor Maritime nieuw Satellite hardware binnenkrijgt moet het getest worden of alle porten op de Satellite werken zoals het ontworpen is. Hiervoor is een testplan en twee applicaties ontwikkeld. Er is als eerst een testplan opgezet dat beschrijft hoe het testen van de hardware werkt. Vervolgens is een applicatie geschreven voor de Satellite. Deze applicatie kan gedraaid worden met een preprocessor definitie. Uiteindelijk is er nog een desktopapplicatie ontwikkeld waarmee de resultaten van het testplan gezien kunnen worden. \newline

\noindent  Uit dit onderzoek is dus gebleken als Satellite gebruikt met verschillende ontwerp technieken een modulair softwaresysteem ontwikkeld worden voor de hardware. In het huidige ontwerp is er ruimte voor nieuwe toevoegingen met zowel sensoren, transformatica en communicatie. Vervolgens kan er makkelijk gewisseld worden dus test applicatie en de Satellite applicatie. Dit helpt Sensor Maritime om continu nieuwe toevoegingen te maken aan de Satellite, zonder dat Sensor Maritime grote onderdelen van de applicatie structuur moet aanpassen.




\newpage
\section{Aanbevelingen}
Tijdens de afstudeerstage was het doel om een modulair softwaresysteem te ontwikkelen voor het bestaand Satellite hardware. Hiervoor is een basis opgezet Tijdens het onderzoeken en implementeren van dit modulair softwaresysteem zijn niet alle functionaliteiten geimplementeerd