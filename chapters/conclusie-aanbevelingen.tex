\chapter{Conclusie en aanbevelingen}
\section{Conclusie}
In dit eindverslag is een onderzoek gedaan naar de volgende onderzoeksvraag: \textbf{hoe kan een modulair softwaresysteem worden ontwikkeld voor de Satellite hardware zodat verschillende sensoren ondersteund worden en data verstuurd kan worden naar één hoofdsysteem?} Omdat de onderzoeksvraag uit twee onderdelen bestaat, namelijk generiek applicatie ontwerp en communicatie met één hoofdsysteem. Worden de bevindingen per onderdeel nader uitgelegd worden. Vervolgens is er ook gekeken worden naar een doelstelling die niet in het onderzoeksvraag staat, maar wat wel een belangrijk onderdeel is van het uiteindelijke product. \newline

\noindent Uit de resultaten van het onderzoek naar een generieke applicatie ontwerp is naar boven gekomen dat het project het best opgezet kan worden met een lagenstructuur, en elke laag heeft een specifieke taak. Met de lagenstructuur kunnen er makkelijk aanpassingen gemaakt worden op specifieke punten zonder dat de applicatielogica aangepast moet worden. Het doel is dat applicatielogica zo min mogelijk aangepast moet worden, dit helpt met robuustheid. Naast de lagenstructuur is er ook gebruik gemaakt van de preprocessor, de preprocessor is een onderdeel van de C taal wat veel gebruikt is in het ontwerpen en ontwikkelen van een generiek applicatie ontwerp. De preprocessor is op het moment zo opgezet dat makkelijk sensoren uit en aangezet kunnen worden. Dit wordt gedaan door definities uit of aan te zetten. Met de preprocessor kunnen dan makkelijk sensoren uit of aangezet worden, maar ook complete stukken logica. \newline

\noindent Het volgende resultaat van het onderzoek naar een robuuste communicatie. Er is hiervoor gefocust op de CAN-communicatie. Met de Satellite is er een probleem ontstaan met het bestaande CAN-protocol van Sensor Maritime. Het huidige protocol ondersteunde niet meerdere sensor data in één bericht. Om dit te ondersteunen is de payload aangepast van het protocol. Hiervoor zijn drie ontwerpen gemaakt en vergeleken. Uiteindelijk is een ontwerp geïmplementeerd en zal ook gebruikt worden voor toekomstige producten. Het laatste onderdeel, is een onderdeel wat Sensor Maritime graag wil zien. Deze doelstelling is een manier ontwikkelen om nieuw Satellite hardware te kunnen testen. Als Sensor Maritime nieuwe Satellite hardware binnenkrijgt moet het getest worden of alle porten op de Satellite werken zoals het ontworpen is. Hiervoor is een testplan en twee applicaties ontwikkeld. Er is als eerste een testplan opgezet dat beschrijft hoe het testen van de hardware werkt. Vervolgens is een applicatie geschreven voor de Satellite. Deze applicatie kan gedraaid worden met een preprocessor definitie. Uiteindelijk is er nog een desktopapplicatie ontwikkeld waarmee de resultaten van het testplan gezien kunnen worden. \newline

\noindent  Uit dit onderzoek is gebleken dat aan de hand van het applicatieontwerp waarbij lagen en preprocessor gebruikt worden, dat er een modulair softwaresysteem ontwikkeld kan worden voor de hardware. Zodat in de toekomst de Satellite er ruimte is voor nieuwe toevoegingen met zowel sensoren, datatransformatie's en communicatie. Vervolgens kan er makkelijk gewisseld worden tussen test applicatie en de Satellite applicatie. Dit helpt Sensor Maritime om continu nieuwe toevoegingen te maken aan de Satellite, zonder dat Sensor Maritime grote onderdelen van de applicatie structuur moet aanpassen.




\newpage
\section{Aanbevelingen}
Tijdens de afstudeerstage was het doel om een modulair softwaresysteem te ontwikkelen voor de bestaande Satellite hardware. Hiervoor is een basis opgezet wat de meeste eisen bevat, wat makkelijk in de toekomst uitgebreid kan worden. Er wordt geadviseerd om deze structuur aan te houden bij toekomstige aanpassingen. Door dit te doen blijft de Satellite modulair en makkelijk uitbreidbaar en dit zal in de toekomst tijd besparen. \newline 

\noindent Buiten de structuur zijn er nog een aantal uitbreidingen wat het product completer maakt. Ten eerste, is er nog geen ondersteuning voor IO link sensoren die beschreven zijn in \ref{tab:eisen} aangezien hier te weinig tijd voor was. Hiervoor zullen nog drivers en abstractielaag geschreven moeten worden. Hiervoor wordt ook aangeraden dat er aan de structuur gehouden wordt die beschreven is in het ontwerp. \newline

\noindent Mijn volgende aanbeveling is om een nieuw manier van testen te ontwikkelen van de Satellite applicatie. Dit kan gedaan worden door bijvoorbeeld gebruik te maken van unit testing, integratie testen. Dit zorgt ervoor dat makkelijk onderdelen van de applicatie getest kunnen worden. \newline

\noindent Mijn laatste aanbeveling om de Satellite applicatie te verbeteren is om een generiek en modulair systeem te verzinnen om data te versturen via UDP. Dit is nog niet geïmplementeerd zodat het modulair is, dit betekent dat het versturen van data handmatig bepaald moet worden. Als voorbeeld zou de UDP packet vergelijkbaar opgebouwd kunnen worden als CAN Bus protocol. Dit verlaagt de handeling en helpt met de robuustheid. \newline


\noindent Om de aanbevelingen te concluderen, zullen er een aantal uitbreidingen gemaakt moeten worden op de volgende onderdelen communicatie, sensor en testen. Met deze toevoegingen kan de applicatie in de toekomst onderhoudbaar, uitbreidbaar en robuust blijven.