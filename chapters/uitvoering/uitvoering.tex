\chapter{Uitvoering}
In dit hoofdstuk wordt de uitvoering en het proces beschreven van de afstudeeropdracht. Het project kan opgesplitst worden in een paar grote onderdelen. De uitvoering zal dan ook opgesplitst worden in de volgende fases. Er wordt er gekeken naar structuur, interfacing, communicatie en sensoren. In deze volgorde is dan ook de applicatie ontwikkeld.

\section{Overzicht}
Om een beeld te krijgen wat er gedaan is in de laatste 20 weken tijdens het afstuderen is er een overzicht gemaakt van welke grootte taken er ontwikkeld zijn. Het diagram laat zien wat er ontwikkeld/ontworpen is en welke stappen hij daarvoor genomen heeft om deze taak goed af te ronden. Er zijn hiervoor een paar fases type gekozen, ten eerste zal er gekeken worden naar het ontwerp, interface, sensor data en communicatie.

\begin{table}[h!]
	\centering
	\caption{Fases van de uitvoering}
	\label{tab:UitvoeringOverzicht}
	\begin{tabular}{lp{13cm}}
	\toprule
	\textbf{Fase} & \textbf{Samenvatting} \\ \midrule
	Ontwerp 				& In deze fase is onderzoek gedaan naar generiek software ontwerp en de payload structuur voor de CAN-communicatie. Daarnaast is er ook gekeken naar het hardware validatie. \\
	Interface 				& In de interface fase is er test en debug applicatie ontworpen voor de applicatie en de hardware validatie onderdeel. 	\\
	Communicatie 			& Hier wordt er gekeken naar de UDP-communicatie en CAN-bus communicatie.												\\
	Sensors  				& Hier wordt er gekeken hoe er verschillende sensoren uitgelezen kunnen worden via verschillende protocollen.			\\ \bottomrule
	\end{tabular}
\end{table}

\newpage
\section{Ontwerp}
Het ontwerp van de applicatie is het belangrijkste onderdeel, dit beantwoord namelijk de hoofdvraag en deelvragen een en twee. Dit moet dan ook goed onderzocht worden, zodat in de toekomst Sensor Maritime veel sensoren kan ondersteunen zonder grotere aanpassing aan de applicatie. Dat is het uiteindelijk doel van het ontwerp, hoe kan er met minimale verandering aan de applicatie nieuwe sensoren toegevoegd worden. Met het ontwerp kunnen de volgende fases gebouwd worden op basis van de bedachte structuur. Het ontwerp is een groot onderdeel en combinatie van verschillende onderdelen van de applicatie en dit zal dan ook in dit hoofdstuk opgesplitst worden in verschillende onderdelen.

\subsection{Applicatie}
Om de structuur op te zetten moest er gekeken worden naar welke programmeertaal er gebruikt gaat worden. Dit was dan ook de eerste taak die student op zich nam. Omdat dit goed te kiezen moet er een analyses gemaakt worden op, wat Sensor Maritime standaard gebruikt. Daarnaast moet er ook gekeken worden of er software ontwerpen/design patterns bestaan die een generieke structuur creeëren. De programmeertaal keuze was met de stagebegeleider besproken waarbij, er twee opties ontstaan zijn. De twee opties waren C of C++, Sensor Maritime gebruikt standaard voor de embedded producten C. Sensor Maritime had minder ervaring met C++. \newline

\noindent Met de programmeer taal gekozen is er een ontwerp gemaakt waardoor de applicatie zo veel mogelijk generiek en robuust is. Dit is aan de hand van de applicatie lagen en de preprocessor gedaan. Met de applicatie lagen kan er aanpassingen of toevoegingen gemaakt worden aan de applicatie zonder dat de applicatie logica veranderd wordt. De preprocessor zorgt ervoor dat er gemakkelijk functionaliteiten en sensoren aan of uit gezet kan worden. Met de preprocessor kan er bijvoorbeeld makkelijk gekozen worden welke versie sensor er gebruikt moet worden. De overige sensoren worden dan ook niet mee gecompileerd.

\subsection{CAN}
Het volgende onderdeel van de structuur is de communicatie via de CAN BUS. Hiervoor is al een bestaand protocol die door Sensor Maritime is ontworpen. Alleen ontbrak er nog aan het bestaande CAN BUS Protocol, om robuust en generiek sensor data op te sturen. Hier moest een systeem voor bedacht worden. Hiervoor zijn drie ontwerpen gemaakt. Een simpele structuur, een generiek modulair structuur en een structuur die er tussen zit. Dit is voorgelegd bij het stagebedrijf, hierover is gediscussieerd wat het meest optimale structuur is voor zowel het eindsysteem als de Satellite. Uit deze discussie is de conclusie getrokken dat structuur 3 de ideale structuur is voor de sensoren.

\subsection{Hardware validatie}
Een van de taken die Sensor Maritime opgelegd heeft is om een systeem te bedenken, zodat als Sensor Maritime nieuw Satellite hardware hebben dat ze die makkelijk kunnen testen. De onderdelen die getest moeten worden zijn IO porten van de Satellite. De manier van testen en wat getest moet worden is als eerst besproken met de stagebegeleider. Hiervoor is een testplan ontwikkeld dat voor elke IO port een manier van testen heeft. Aan de hand van het testplan is er een test applicatie geschreven wat de applicatie test. Om alle firmware die voor de Satellite is ontwikkeld is de Satellite applicatie en test applicatie samengevoegd. Hiervoor is de preprocessor gebruikt. Sensor Maritime hoeft alleen maar een definitie uit of aan te zetten en opnieuw compileren. Hiermee kan Sensor Maritime makkelijk wisselen tussen hardware validatie en echte Satellite applicatie.

\section{Interface}
Om de uiteindelijk Satellite applicatie, en testplan applicatie te kunnen testen zal er een applicatie nodig zijn die kijkt of de hardware applicatie wel werkt na toebehoren. Hiervoor zijn twee desktop applicaties ontwikkeld. Deze twee applicaties zullen opgesplitst worden in subhoofdstukken die hieronder beschreven zijn.

\subsection{Satellite debug applicatie}
Er is begonnen met een debug applicatie te schrijven om alle functionaliteiten te testen van de Satellite. Dit is ontwikkeld met C\#. Het doel was dat de Satellite met de desktopapplicatie communiceert. Om de functionaliteiten te testen zijn verschillende visualiseren methodes toegepast per sensor. Een voorbeeld van de visualiseer methode is om een grafiek te laten zien van de actuele waarden te laten zien. 

\subsection{Testplan applicatie}
De testplan applicatie is ontwikkeld om sommige onderdelen van het testplan te testen. Sommige porten zijn lastig om te verifiëren zonder een desktopapplicatie. Het doel van het testplan applicatie is dat Sensor Maritime snel kan zien of de hardware werkt zoals verwacht. Hiervoor is ook C\# gebruikt gemaakt. Voor ontvangen is er een tabel toegevoegd wat de laatste waarden ontvangen en vertoond. Vervolgens is er user input toegevoerd zodat de gebruiker kan communiceren met de Satellite via serieel communicatie.

\section{Communicatie}
De communicatie is een groot onderdeel van de applicatie. Vanuit Satellite moet er gecommuniceerd worden met een eindsysteem. Tijdens de implementatie van de applicatie zijn verschillende vormen van communicatie gebruikt met het eindsysteem. Deze twee vormen van communicatie zijn bepaald door de klant. \newline

Om een robuuste communicatie op te zetten zijn er verschillende stappen gezet voor zowel UDP als voor CAN BUS. Ten eerste is er gekeken hoe het packet opgebouwd kan worden wat uiteindelijk opgestuurd. Met het packet opgebouwd is er gekeken naar foutafhandeling, dit is een belangrijk onderdeel omdat de klant niet wil dat de Satellite crasht tijdens het versturen van data. Hiervoor is gekeken naar foutafhandeling en welke stappen er gezet moet worden bij eventuele fouten.


\section{Sensors}
De volgende fase was om voor de sensoren een abstractie en driver laag ontwikkelen. De abstractie laag zal bepalen hoe de sensor zich moet gedragen. De abstractie laag is het tussenpersoon van de applicatie logica en sensor drivers. Het idee is dan ook dat de abstractie laag met de hulp van preprocessor bepaald welke sensor gebruikt gaat worden. Met de abstractie laag is er een systeem ontwikkeld wat antwoord geeft op deelvraag 2. Voor elke sensor is een nieuwe definitie toegevoegd wat bepaald of de sensor abstractie laag de sensor driver aanroept of niet. Er zijn verschillende type sensoren, sommige sensoren werken via SPI, serieel of juist analoog. Dit wordt allemaal uitgewerkt in de sensor driver. Elke sensor is anders in werking, dus drivers zal altijd toegevoegd worden.

