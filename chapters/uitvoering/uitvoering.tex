\chapter{Uitvoering}
In dit hoofdstuk worden de uitvoering en het proces beschreven van de afstudeeropdracht. Het project kan opgesplitst worden in een paar grote onderdelen. De uitvoering zal dan ook opgesplitst worden in de volgende fases. Er wordt gekeken naar structuur, interfacing, communicatie en sensoren. In deze volgorde is dan ook de applicatie ontwikkeld.

\section{Overzicht}
Om een beeld te krijgen van wat er gedaan is in de 20 weken van de afstudeerstage is een overzicht gemaakt van welke grote taken er uitgevoerd zijn. Het diagram laat zien wat er ontwikkeld/ontworpen is en welke stappen daarvoor genomen zijn om deze taak goed af te ronden. Er zijn hiervoor een paar type van fases gekozen: ontwerp, interface, sensor data en communicatie.

\begin{table}[h!]
	\centering
	\caption{Fases van de uitvoering}
	\label{tab:UitvoeringOverzicht}
	\begin{tabular}{lp{13cm}}
	\toprule
	\textbf{Fase} & \textbf{Samenvatting} \\ \midrule
	Ontwerp 				& In deze fase is onderzoek gedaan naar het generiek software ontwerp en de payload structuur voor de CAN-communicatie. Daarnaast is gekeken naar de hardware validatie. \\
	Interface 				& In de interface fase is test- en debug applicatie ontworpen voor de eigenlijke applicatie en het hardware validatie onderdeel. 	\\
	Communicatie 			& Hier is gekeken naar de UDP-communicatie en CAN-bus communicatie.												\\
	Sensors  				& Hier is onderzocht hoe verschillende sensoren uitgelezen kunnen worden via verschillende protocollen.			\\ \bottomrule
	\end{tabular}
\end{table}

\newpage
\section{Ontwerp}
Het ontwerp van de applicatie is het belangrijkste onderdeel. Dit beantwoordt namelijk de hoofdvraag en deelvragen 1 en 2. Dit vereist daarom goed onderzoek, zodat in de toekomst Sensor Maritime veel sensoren kan ondersteunen zonder grote aanpassingen aan de applicatie. Dat is het uiteindelijke doel van het ontwerp is vast te stellen hoe met minimale verandering aan de applicatie nieuwe sensoren toegevoegd kunnen worden. Vanuit het ontwerp kunnen de volgende fases gebouwd worden op basis van de bedachte structuur. Het ontwerp is een groot onderdeel en een combinatie van verschillende onderdelen van de applicatie en dit zal dan ook in dit hoofdstuk besproken worden in verschillende paragrafen.

\subsection{Applicatie}
Om de structuur op te zetten moest er gekeken naar de te gebruiken programmeertaal. Dit was dan ook de eerste taak die student op zich nam. Om een goede keuze te maken is een analyse gemaakt worden na de standaard die Sensor Maritime gebruikt. Daarnaast is gekeken of er software ontwerpen/design patterns bestaan die een generieke structuur creeëren. De programmeertaal keuze is met de stagebegeleider besproken waarbij twee opties naar voor zijn gekomen. De twee opties zijn C of C++. Sensor Maritime gebruikt standaard voor de embedded producten C. Sensor Maritime heeft minder ervaring met C++. \newline

\noindent Met de gekozen programmeertaal is een ontwerp gemaakt waardoor de applicatie zo veel mogelijk generiek en robuust is. Dit is aan de hand van de applicatielagen en de preprocessor gedaan. Met de applicatielagen kunnen er aanpassingen of toevoegingen gemaakt worden aan de applicatie zonder dat de applicatielogica veranderd wordt. De preprocessor zorgt ervoor dat gemakkelijk functionaliteiten en sensoren aan of uitgezet kunnen worden. Met de preprocessor kan bijvoorbeeld makkelijk gekozen worden welke versie sensor gebruikt moet worden. De overige sensoren worden dan niet mee gecompileerd.

\subsection{CAN}
Het volgende onderdeel van de structuur is de communicatie via de CAN BUS. Hiervoor is een al bestaand protocol dat door Sensor Maritime is ontworpen. Alleen ontbrak aan het bestaande CAN BUS protocol de mogelijkheid, om robuust en generiek sensor data op te sturen. Hier moest een systeem voor bedacht worden. Hiervoor zijn drie ontwerpen gemaakt. Een simpele structuur, een generiek modulaire structuur en een structuur die ertussen zit. Dit is voorgelegd bij het afstudeerbedrijf, hierover is gediscussieerd wat de meest optimale structuur is voor zowel het eindsysteem als de Satellite. Uit deze discussie is de conclusie getrokken dat structuur 3 de ideale structuur is voor de sensoren.

\subsection{Hardware validatie}
Een van de taken die Sensor Maritime opgelegd heeft is om een systeem te bedenken, zodat als Sensor Maritime nieuw Satellite hardware heeft, ze die makkelijk kunnen testen. De onderdelen die getest moeten worden zijn IO porten van de Satellite. De manier van testen en wat getest moet worden is als eerste besproken met de stagebegeleider. Hiervoor is een testplan ontwikkeld dat voor elke IO port een manier van testen heeft. Aan de hand van het testplan is een testapplicatie geschreven. De firmware van de Satellite heeft twee applicaties die samengevoegd zijn. De firmware bestaat uit de Satellite applicatie en de test applicatie. Hiervoor is de preprocessor gebruikt. Sensor Maritime hoeft alleen maar een definitie uit of aan te zetten en opnieuw compileren. Hiermee kan Sensor Maritime makkelijk wisselen tussen hardware validatie en echte Satellite applicatie.

\section{Interface}
Om de uiteindelijke Satellite applicatie, en testplan applicatie te kunnen testen zal er een applicatie nodig zijn die kijkt of de hardware applicatie wel werkt naar behoren werkt. Hiervoor zijn twee desktopapplicaties gemaakt. Deze twee applicaties zijn hierna in subhoofdstukken beschreven.

\subsection{Satellite debug applicatie}
Er is begonnen met een debug applicatie te schrijven om alle functionaliteiten te testen van de Satellite. Deze is ontwikkeld in C\#. Het doel is dat de Satellite met de desktopapplicatie communiceert. Om de functionaliteiten te testen zijn verschillende visualisatie methodes toegepast per sensor. Een voorbeeld van de visualisatie methode is om een grafiek te laten zien van de actuele waarden. 

\subsection{Testplan applicatie}
De testplan applicatie is ontwikkeld om sommige onderdelen van het testplan uit te voeren. Bepaalde porten zijn lastig om te verifiëren zonder een desktopapplicatie. Het doel van de testplan applicatie is dat Sensor Maritime snel kan zien of de hardware werkt zoals verwacht. Hiervoor is ook van C\# gebruikt gemaakt. Er is een tabel toegevoegd wat de laatst ontvangen waarden ontvangen en vertoond. Vervolgens is er user input toegevoegd, zodat de gebruiker kan communiceren met de Satellite via seriële communicatie.

\section{Communicatie}
De communicatie is een groot onderdeel van de applicatie. Vanuit Satellite moet er gecommuniceerd worden met een eindsysteem. Tijdens de implementatie van de applicatie zijn verschillende vormen van communicatie gebruikt met het eindsysteem. Deze vormen van communicatie zijn bepaald door de klant. \newline

\noindent Om een robuuste communicatie op te zetten zijn er verschillende stappen gezet voor zowel UDP als voor CAN BUS. Ten eerste is gekeken hoe het packet wat uiteindelijk opgestuurd wordt, opgebouwd kan worden. Met het packet opgebouwde packet is gekeken naar foutafhandeling. Dit is een belangrijk onderdeel omdat de klant niet wil dat de Satellite crasht tijdens het versturen van data. Hiervoor is gekeken naar foutafhandeling en welke stappen gezet moet worden bij eventuele fouten.


\section{Sensors}
De volgende fase is om voor de sensoren een abstractielaag en driverlaag te ontwikkelen. De abstractielaag zal bepalen hoe de sensor zich moet gedragen. De abstractielaag is de tussenpersoon tussen de applicatielogica en sensor drivers. Het idee is dan ook dat de abstractielaag met de hulp van de preprocessor bepaalt welke sensor gebruikt gaat worden. Met de abstractielaag is een systeem ontwikkeld wat antwoord geeft op deelvraag 2. Voor elke sensor is een nieuwe definitie toegevoegd wat bepaalt of de sensor abstractielaag de sensor driver aanroept of niet. Er zijn verschillende type sensoren, sommige sensoren werken via SPI, serieel of juist analoog. Dit wordt allemaal uitgewerkt in de sensor driver. Elke sensor is anders in werking, dus drivers zullen altijd toegevoegd worden.