\chapter{Resultaat NOT DONE}
In dit hoofdstuk zullen alle resultaten van de afstudeeropdracht beschreven worden. De resultaten zullen opgesplitst zijn in twee onderdelen. Ten eerste zal er teruggekoppeld worden op de doelstellingen en eisen. Ten tweede zal er gekeken worden of de Satellite voldoet aan het opgezette hardware testplan.

\section{Doelstellingen}
In het begin van het project zijn er drie doelstellingen gedefineerd, de doelstellingen hebben als taak om een overzicht te krijgen wat er precies bereikt zal moeten worden. De volgende doelstellingen zijn hieronder gedefineerd.
\begin{enumerate}
	\item De Satellite IO porten zijn volledig getest, en kunnen communiceren met sensoren.
	\item De Satellite heeft een robuuste communicatie over CAN en UDP.
	\item Generiek en modulair opgebouwd software, waardoor het in de toekomst makkelijk uitgebreid kan worden.
\end{enumerate}


\section{Hardware validatie testplan}
In het begin is er een testplan ontwikkeld dat de IO porten van de Satellite test. Met deze testplan kan nieuwe hardware snel en makkelijk getest worden. Het resultaat is dat testplan ingevuld is voor het huidige hardware. In tabel is het verwachte resultaat beschreven en het uiteindelijk resultaat.

\subsection{DIP Switch}
Hieronder is het resultaat \ref{tab:resultaatdip} van het testplan dat gaat over de DIP switches. 
\begin{table}[h!]
	\centering
	\caption{Resultaat testplan van de DIP switches}
	\begin{tabular}{lp{8.5cm}lp{4cm}}
	\toprule
	\textbf{Naam} 	& \textbf{Verwachte resultaat} & \textbf{Geslaagd} & \textbf{Opmerking}\\ \toprule
	PU1				& Als het aan gezet wordt de range van de analoge input 1 verhoogt naar 24 volt, dit verhoogt de waarde wat de microcontroller leest, de waarde wordt laten zien op een GUI. &\\
	-				& Wordt niet getest. & \\ 
	PU2				& Als het aan gezet wordt de range van de analoge input 2 verhoogt naar 24 volt, dit verhoogt de waarde wat de microcontroller leest, de waarde wordt laten zien op een GUI. &\\
	PU3				& Als het aan gezet wordt de range van de analoge input 3 verhoogt  naar 24 volt, dit verhoogt de waarde wat de microcontroller leest, de waarde wordt laten zien op een GUI. & \\
	ADD0 			& Zet 3.3V op de pin van de microcontroller, als de microcontroller 3.3V detecteert zet het led 0 aan. & \\
	ADD1 			& Zet 3.3V op de pin van de microcontroller, als de microcontroller 3.3V detecteert zet het led 1 aan. & \\
	ADD2 			& Zet 3.3V op de pin van de microcontroller, als de microcontroller 3.3V detecteert zet het led 2 aan. & \\
	ADD3 			& Zet 3.3V op de pin van de microcontroller, als de microcontroller 3.3V detecteert zet het led 3 aan. & \\ \bottomrule
	\end{tabular}
	\label{tab:resultaatdip}
\end{table}

\newpage
\subsection{Digital output signaal}
Hieronder is het resultaat \ref{tab:resultaatdio} van het testplan dat gaat over de digitale output signalen.
\begin{table}[h!]
	\centering
	\caption{Resultaat testplan van de digital output signalen}
	\begin{tabular}{lp{8.5cm}lp{4cm}}
	\toprule
	\textbf{Naam} 	& \textbf{Verwachte resultaat} & \textbf{Geslaagd} & \textbf{Opmerking} \\ \toprule
	D1	&	Als op de pin 24V staat gaat het lampje aan, en bij 0V uit gaat het uit. Dit gebeurt om 1 seconden. && \\			
	D2	&	Als op de pin 24V staat gaat het lampje aan, en bij 0V uit gaat het uit. Dit gebeurt om 1 seconden. && \\			
	D3	&	Als op de pin 24V staat gaat het lampje aan, en bij 0V uit gaat het uit. Dit gebeurt om 1 seconden. && \\			
	D4	&	Als op de pin 24V staat gaat het lampje aan, en bij 0V uit gaat het uit. Dit gebeurt om 1 seconden. && \\ \bottomrule
	\end{tabular}
	\label{tab:resultaatdio}
\end{table}

\subsection{Analoog input signaal}
Hieronder is het resultaat \ref{tab:resultaatai} van het testplan dat gaat over de analoog input signalen.
\begin{table}[h!]
	\caption{Resultaat testplan van de analoog input signalen}
	\begin{tabular}{lp{8.5cm}lp{4cm}}
	\toprule
	\textbf{Naam} 	& \textbf{Verwachte resultaat} & \textbf{Geslaagd} & \textbf{Opmerking}\\ \toprule
	AI1			& Als de schakelaar PU1 van de schakelaar aangezet wordt dan zal de data bereik op de graphical user interface aangepast worden naar 0 tot 4096. Als het dip schakelaar uit staat dan zal de bereik maar van 0 tot 1100 gaan.\\
	AI2			& Als de schakelaar PU1 van de schakelaar aangezet wordt dan zal de data bereik op de graphical user interface aangepast worden naar 0 tot 4096. Als het dip schakelaar uit staat dan zal de bereik maar van 0 tot 1100 gaan.\\
	AI3			& Als de schakelaar PU1 van de schakelaar aangezet wordt dan zal de data bereik op de graphical user interface aangepast worden naar 0 tot 4096. Als het dip schakelaar uit staat dan zal de bereik maar van 0 tot 1100 gaan.\\  \bottomrule
	\end{tabular}
	\label{tab:resultaatai}
\end{table}

\newpage
\subsection{Relay}
Hieronder is het resultaat \ref{tab:resultaatrelay}  van het testplan dat gaat over de relay.
\begin{table}[h!]
	\caption{Resultaat testplan van de relay}
	\begin{tabular}{lp{8.5cm}lp{4cm}}
	\toprule
	\textbf{Naam} 	& \textbf{Verwachte resultaat} & \textbf{Geslaagd} & \textbf{Opmerking} \\ \toprule
	Relay			& De microcontroller zet de pin van de relay voor vijf seconden hoog, als dit gebeurt hoor je een klik te horen. Na vijf seconden zal deze pin weer laag gezet worden, dit moet ook gehoord kunnen worden.\\  \bottomrule
	\end{tabular}
	\label{tab:resultaatrelay}
\end{table}


\subsection{Serieële communicatie}
Hieronder is het resultaat \ref{tab:resultaatserieel} van het testplan dat gaat over de serieële communicatie.
\begin{table}[h!]
	\caption{Resultaat testplan van de serieële communicatie}
	\begin{tabular}{lp{8.5cm}lp{4cm}}
	\toprule
	\textbf{Naam} 	& \textbf{Verwachte resultaat} & \textbf{Geslaagd} & \textbf{Opmerking} \\ \toprule
	UART & Gebruiker stuurt de een bericht op en de Satellite antwoord hierop met het bericht wat was ontvangen is.&& \\
	UART 2 & Gebruiker stuurt de een bericht op en de Satellite antwoord hierop met het bericht wat was ontvangen is.&& \\
	RS232 & Gebruiker stuurt de een bericht op en de Satellite antwoord hierop met het bericht wat was ontvangen is.&&\\
	RS422 & Gebruiker stuurt de een bericht op en de Satellite antwoord hierop met het bericht wat was ontvangen is.&&\\ \bottomrule
	\end{tabular}
	\label{tab:resultaatserieel}
\end{table}

