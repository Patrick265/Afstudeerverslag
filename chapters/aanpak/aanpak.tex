\chapter{Aanpak} \label{ch:aanpak}

\section{Projectactiviteiten}
De afstudeerderstage is opgesplitst in verschillende fases, in dit hoofdstuk worden er voor elke fase de activiteiten geformuleerd. In de stage zijn er drie fases gespecificeerd, ten eerste zal er gekeken worden naar het onderzoek, de implementatie, en uiteindelijk wordt het project afgerond. In de onderzoek fases wordt de criteria concreet gemaakt en wordt er gekeken naar de beste mogelijke implementatie van het gevraagde product. Na het onderzoek komt de ontwerp en implementatie fase, hier wordt het uiteindelijke product geïmplementeerd aan de hand van de onderzoek en criteria die in de vorige fase zijn opgesteld. Als de implementatie voltooid is komt de laatste fase, de afronding. In de afronding wordt er gekeken naar het uiteindelijke product aan de hand van de criteria. Voor elke fase zal er verteld worden wat de activiteiten zijn en het resultaat van deze activiteiten.

\subsection{Onderzoek}
Tijdens de onderzoeksfase zal er een aantal stappen gemaakt worden om de het project af te bakenen en er wordt gekeken naar de implementatie van het project. Tijdens de stage komen de volgende activiteiten aanbod:
\begin{enumerate}
	\item Een onderzoek doen naar een generiek applicatie ontwikkelen, waarbij verschillende sensoren ondersteund kunnen worden.
	\item Een onderzoek doen naar payload ontwerp van de CAN-bus communicatie.
	\item Een onderzoek doen naar hardware validatie.
	\item Plan van aanpak
\end{enumerate}

\subsection{Implementatie}
In de implementatie fase worden de resultaten van de onderzoeksfase verwerkt in de applicatie. 
\begin{enumerate}
	\item Implementatie van het generieke software ontwerp
	\item Implementatie van de hardware validatie aan de hand van het testplan.
	\item Implementatie van het payload structuur van de CAN-bus communicatie.
\end{enumerate}

\newpage
\subsection{Afronding}
De afronding is de laatste fase van het project tijdens de afstudeerstage. In deze fase wordt er verwacht dat het product robuust en stabiel is en compleet genoeg om een demo te geven. In deze fase wordt de applicatie helemaal getest op eventuele bugs/glitches of incorrecte afhandeling van fouten. Daarnaast wordt er ook verwacht dat het eindverslag af voltooid is.
\begin{enumerate}
	\item Verwerken van het afstudeerverslag.
	\item De software wordt verder getest op eventuele bugs en glitches.
	\item Hardware validatie applicatie wordt vergeleken met het testplan.
\end{enumerate}


\section{Tussenresultaten}
Dit hoofdstuk laat zien de tussenresultaten van het project. De tussenresultaten wordt opgesplitst tussen milestones en documenten.

\subsection{Milestones}
Hieronder is een tabel voor de milestones \ref{tab:milestones}, de milestones zijn niet gelinkt aan documenten maar juist de implementatie van het uiteindelijke product.
\begin{table}[h!]
		\caption{Milestones}
		\begin{tabular}{p{1cm}p{14cm}}
		\toprule
		\textbf{\#} & \textbf{Beschrijving} \\ \midrule
		1 & Software ontwerp om de generieke sensoren te ondersteunen      \\
		2 & Payload structuur ontwerp voor het CAN Bus protocol      \\
		3 & Implementatie applicatie aan de hand van het software ontwerp					\\
		4 & Hardware validatie geïmplementeerd		\\
		5 & Afronding ongeteste product \\
		6 & Afronding product              \\ \bottomrule
		\end{tabular}

	\label{tab:milestones}
\end{table}

\subsection{Documenten}
Hieronder is een tabel \ref{tab:documents} dat alle documenten laat zien wat door de stagiaire wordt gemaakt, en welke producten aan de Sensor Maritime geleverd wordt.
\begin{table}[h!] 
	\caption{Tussenresultaten van documenten}
	\begin{tabular}{p{1cm}p{14cm}}
	\toprule
	\textbf{\#} & \textbf{Documentnaam}   \\ \midrule
	1 & Definitief Plan van Aanpak afstuderen. \\
	2 & Onderzoeksdocument naar een generieke methode om sensoren te ondersteunen. \\
	3 & Document voor het payload structuur ontwerp voor het bestaande CAN Bus protocol \\
	4 & Testplan hardware validatie \\
	5 & Eindverslag afstuderen \\ \bottomrule
	\end{tabular}

\label{tab:documents}
\end{table}

\newpage
\section{Kwaliteit}
In dit hoofdstuk wordt er gekeken hoe de kwaliteit gewaarborgd wordt tijdens het project. Kwaliteit wordt onderverdeeld in verschillende hoofdstukken: kwaliteit van het product en kwaliteit van de documentatie. Er wordt gekeken hoe de kwaliteit van het product, het best gewaarborgd kan worden. Daarnaast verzekert dit ook naar een succesvolle afronding van de stage.

\subsection{Product}
De kwaliteit wordt bepaald door de klant en bedrijfsbegeleider. De afstudeerder is verantwoordelijk voor het product en zal de requirements opzetten met de bedrijfsbegeleider en klant. Met de requirements kunnen de milestones gedefinieerd worden. Met de hulp van de eisen en milestones wordt het project meetbaar en weet wat de klant uiteindelijk kan verwachten. \newline

\noindent Tijdens de implementatie zal er gebruikt gemaakt worden van GitHub voor versiebeheer. Sensor Maritime heeft een eigen GitHub account waar alle projecten van Sensor Maritime opstaat. Met de hulp van Github kan er makkelijk gekeken worden hoe het project verloopt en code kan beoordeeld worden door de bedrijfsbegeleider. Op Github zal een uiteindelijke release gemaakt voor prototypes. Aan het einde van de stage zal een final release gemaakt worden waar alles toegevoegd wordt. Hier komt alle documentatie en informatie te staan om het project te kunnen bouwen en te uploaden naar het hardware. In de final release moet het product minimaal voldoen aan de must have eisen van de MoSCoW analyse. Mocht dit niet zijn dan zou het betekenen dat het product niet goedkeurt is. Bij eventuele problemen kan de stagiair ook hulp vragen van de bedrijfsbegeleider om zo de kwaliteit te waarborgen. Er zal geen hulp komen van partijen die niet in het plan van aanpak zijn genoemd. Dit wordt aan het einde van de stage bekeken met de klant en de bedrijfsbegeleider.

\subsection{Documentatie}
Net zoals het product zal de stagiair alle verantwoordelijkheid hebben voor de documentatie dat gemaakt is door de stagiair. Alle documenten worden in fases opgeleverd. Het conceptdocumenten zal bekeken worden door de bedrijfsbegeleider. De bedrijfsbegeleider zal hier opmerkingen geven en de stagiair zal deze opmerking verwerken voordat het opgestuurd wordt naar de docentbegeleider, de docentbegeleider is het laatste persoon die het document zal bekijken, de student zal alle opmerking verwerken wat een definitief document geeft. Alle versies van het documenten zal opgeslagen worden en zal uiteindelijk op OneDrive gezet worden volgens de richtlijnen van Sensor Maritime. De bedrijfsdocumenten worden niet in fases opgeleverd, de eerste versie van het bedrijfsdocument zal altijd een conceptverslag zijn.