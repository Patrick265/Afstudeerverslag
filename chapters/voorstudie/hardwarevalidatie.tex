\subsection{Inleiding}
Elke nieuwe hardware wat aan Satellite wordt geleverd zal door Sensor Maritime moeten gevalideerd moeten worden dit betekend dat hiervoor een testplan ontwikkeld moet worden. Er zal als eerst beschreven worden welke porten er getest moet worden en hoe deze getest gaan worden en de uiteindelijke verwachte resultaat. De volgende type porten worden besproken. Ten eerste zal er gekeken worden naar DIP schakelaars, digitale output, analoge input, relay en uiteindelijk seriele communicatie. \newline

\noindent De Satellite heeft verschilllende porten porten zijn die gebruikt kan worden. Elke type port wordt beschreven in de volgende hoofdstukken. In deze hoofdstukken wordt beschreven wat de port moet doen en hoe dit getest gaat worden en uiteindelijk wat het verwachte resultaat is.

\subsection{DIP switch}
\subsubsection{Porten}
Een DIP switch bestaat uit een aantal schakelaars die je kan of uit kan zitten \autocite{DIP}. Op de Satellite is er een DIP switch aan bord, die acht schakelaar is heeft. Zeven van de acht worden op het moment gebruikt. Het volgende tabel \ref{tab:hw_val_dip} geeft aan wat de acties zijn als je de schakelaar aan of uit zet. De eerste drie van de schakelaars worden gebruikt als configuratie voor andere porten die worden beschreven in hoofdstuk \ref{Analog Input Signaal}. De laatste vier schakelaars zetten een specifiek pin van de microcontroller hoog of laag.
\begin{table}[h!]
	\caption{DIP switch porten die gevalideerd moeten worden}
	\begin{tabular}{llllp{10cm}}
	\toprule
	\textbf{Naam} & \textbf{GPIO} & \textbf{Pin} & \textbf{IO} & \textbf{Beschrijving}	\\ \toprule
	PU1		& -			& - 	& -    		& Verandert de analog input 1 voltage tussen 0 en 24V	\\
	-		& -			& - 	& -    		& Wordt niet gebruikt.								\\
	PU2		& -			& - 	& -    		& Verandert de analog input 2 voltage tussen 0 en 24V	\\
	PU3		& -			& - 	& -    		& Verandert de analog input 3 voltage tussen 0 en 24V	\\
	ADD0 	& 3			& 10	& Input		& Zet de pin hoog of laag, aan de hand van of het aan staat of niet.		\\
	ADD1 	& 3			& 11	& Input		& Zet de pin hoog of laag, aan de hand van of het aan staat of niet.		\\
	ADD2 	& 3			& 12	& Input		& Zet de pin hoog of laag, aan de hand van of het aan staat of niet.		\\
	ADD3 	& 3			& 0 	& Input		& Zet de pin hoog of laag, aan de hand van of het aan staat of niet.		\\ \bottomrule
	\end{tabular}
	\label{tab:hw_val_dip}
\end{table}
\newpage

\subsubsection{Testplan}
De eerste drie schakelaars worden anders getest in vergelijking met de laatste vier. De laatste vier schakelaars maken vier leds branden die op Satellite zitten. In het volgende tabel \ref{tab:hw_val_dip_testplan} is een overzicht van het testplan:
\begin{table}[h!]
	\caption{DIP switch porten testplan}
	\begin{tabular}{lp{14.5cm}}
	\toprule
	\textbf{Naam} 	& \textbf{Verwachte resultaat} \\ \toprule
	PU1				& Als het aan gezet wordt de range van de analoge input 1 verhoogt naar 24 volt, dit verhoogt de waarde wat de microcontroller leest, de waarde wordt laten zien op een GUI.\\
	-				& Wordt niet getest. \\
	PU2				& Als het aan gezet wordt de range van de analoge input 2 verhoogt naar 24 volt, dit verhoogt de waarde wat de microcontroller leest, de waarde wordt laten zien op een GUI.\\
	PU3				& Als het aan gezet wordt de range van de analoge input 3 verhoogt  naar 24 volt, dit verhoogt de waarde wat de microcontroller leest, de waarde wordt laten zien op een GUI. \\
	ADD0 			& Zet 3.3V op de pin van de microcontroller, als de microcontroller 3.3V detecteert zet het led 0 aan.\\
	ADD1 			& Zet 3.3V op de pin van de microcontroller, als de microcontroller 3.3V detecteert zet het led 1 aan.\\
	ADD2 			& Zet 3.3V op de pin van de microcontroller, als de microcontroller 3.3V detecteert zet het led 2 aan.\\
	ADD3 			& Zet 3.3V op de pin van de microcontroller, als de microcontroller 3.3V detecteert zet het led 3 aan.\\ \bottomrule
	\end{tabular}
	\label{tab:hw_val_dip_testplan}
\end{table}

\newpage
\subsection{Digital output signaal}
De digitale output signaal kan alleen hoog of uit gezet worden, dit betekent dat het geen input modus heeft. De Satellite heeft vier digitale output porten die op 24V of 0V gezet kan worden. Het volgende tabel geeft een overzicht \ref{tab:hw_val_dio}.
\subsubsection{Porten}
\begin{table}[h!]
	\caption{Digital output signalen die gevalideerd moeten worden}
	\begin{tabular}{llllp{10cm}}
	\toprule
	\textbf{Naam} & \textbf{GPIO} & \textbf{Pin} & \textbf{IO} & \textbf{Beschrijving}				 	\\ \toprule
	D1			& 1			& 6    	& Output	& Digital output signaal, op deze pin komt 24V of 0V. \\
	D2			& 1			& 7    	& Output	& Digital output signaal, op deze pin komt 24V of 0V. \\
	D3			& 3			& 3    	& Output	& Digital output signaal, op deze pin komt 24V of 0V. \\
	D4			& 3			& 2   	& Output	& Digital output signaal, op deze pin komt 24V of 0V. \\ \bottomrule
	\end{tabular}
	\label{tab:hw_val_dio}
\end{table}

\subsubsection{Testplan}
Alle digitale output porten voeren dezelfde actie. Het doel is dat als een LED verbonden is met de port dat die dan aan of uit gaat om de 1 seconden.
\begin{table}[h!]
	\caption{Digital output signalen testplan}
	\begin{tabular}{lp{14.5cm}}
	\toprule
	\textbf{Naam} 	& \textbf{Verwachte resultaat} \\ \toprule
	D1	&	Als op de pin 24V staat gaat het lampje aan, en bij 0V uit gaat het uit. Dit gebeurt om 1 seconden. \\			
	D2	&	Als op de pin 24V staat gaat het lampje aan, en bij 0V uit gaat het uit. Dit gebeurt om 1 seconden. \\			
	D3	&	Als op de pin 24V staat gaat het lampje aan, en bij 0V uit gaat het uit. Dit gebeurt om 1 seconden. \\			
	D4	&	Als op de pin 24V staat gaat het lampje aan, en bij 0V uit gaat het uit. Dit gebeurt om 1 seconden. \\ \bottomrule
	\end{tabular}
	\label{tab:hw_val_dio_testplan}
\end{table}

\newpage
\subsection{Analog input signaal} \label{Analog Input Signaal}
\subsubsection{Porten}
De analog signalen kan alleen uitgelezen worden, dit betekent dat het alleen een input modus heeft. De Satellite heeft drie analoge input porten aan bord. In het volgende tabel wordt een overzicht gegeven \ref{tab:hw_val_ai}.
\begin{table}[h!]
	\caption{Analog input signalen die gevalideerd moeten worden}
	\begin{tabular}{llllp{9cm}}
	\toprule
	\textbf{IO} & \textbf{GPIO} & \textbf{Pin} & \textbf{Input/Output} & \textbf{Beschrijving}			\\ \toprule
	AI1			& 14		& 2    	& Input		& Analoge input signaal van 0 tot 24 volt.					\\
	AI2			& 14		& 3    	& Input		& Analoge input signaal van 0 tot 24 volt.					\\
	AI3			& 14		& 12   	& Input		& Analoge input signaal van 0 tot 24 volt.					\\  \bottomrule
	\end{tabular}
	\label{tab:hw_val_ai}
\end{table}
\subsubsection{Testplan}
Aan de hand wat is beschreven zal er gebruik gemaakt worden van de DIP schakelaars PU1, PU2, en PU3 om de analoge input te testen. De microcontroller heeft een 12 bit Analog to Digital Converter (ADC) \autocite{microcontroller}. Dit betekent dat het bereik van analoge input porten van 0 tot 4096 ($2^{12}$) is. Dus bij 24 volt zal de microcontroller ongeveer 4096 in lezen, het getal kan af en toe lager zijn in ver band met ruis. Bij 0 volt zal de microcontroller 0 in lezen. PU1, PU2, PU3 zet het analoge bereik tussen 3,3 volt en 24 volt. Dit betekent als PU1, PU2 of PU3 uitstaan dat de maximale bereik 3,3 volt is en als het aan staat is de bereik 24 Volt. Om het bereik te testen zal er gebruikt gemaakt worden van een Graphical User Interface (GUI). De microcontroller zal om 100 microseconden de analoge input omzetten naar een digitale waarden en dit opsturen over UDP. Hiermee kan geverifieerd worden dat als de DIP schakelaars uitstaan dat het bereik van de waarden van 0 tot 1100 gaan. Als de DIP schakelaars aan staan dan zal het bereik zijn van 0 tot 4096. Hieronder is een overzicht van hoe de analoge input porten gevalideerd worden \ref{tab:hw_val_ai_testplan}.
\begin{table}[h!]
	\caption{Testplan voor de analoge input signalen}
	\begin{tabular}{lp{14.5cm}}
	\toprule
	\textbf{Naam} 	& \textbf{Verwachte resultaat} \\ \toprule
	AI1			& Als de schakelaar PU1 van de schakelaar aangezet wordt dan zal de data bereik op de graphical user interface aangepast worden naar 0 tot 4096. Als het dip schakelaar uit staat dan zal de bereik maar van 0 tot 1100 gaan.\\
	AI2			& Als de schakelaar PU1 van de schakelaar aangezet wordt dan zal de data bereik op de graphical user interface aangepast worden naar 0 tot 4096. Als het dip schakelaar uit staat dan zal de bereik maar van 0 tot 1100 gaan.\\
	AI3			& Als de schakelaar PU1 van de schakelaar aangezet wordt dan zal de data bereik op de graphical user interface aangepast worden naar 0 tot 4096. Als het dip schakelaar uit staat dan zal de bereik maar van 0 tot 1100 gaan.\\  \bottomrule
	\end{tabular}
	\label{tab:hw_val_ai_testplan}
\end{table}


\subsection{Relay}
\subsubsection{Porten}
Er is een relay verbonden aan de microcontroller. Een relay is een mechanisch apparaat dat een elektrische connectie maakt tussen twee of meer punten op reactie van de microcontroller. Als er 3,3 volt op de relay gezet wordt zal de relay sluiten en bij 0 volt zal het weer openen gaan. Omdat het mechanisch is moet je dit ook kunnen horen \autocite{relay}. Hieronder is een overzicht van de relay \ref{tab:hw_val_relay}.

\begin{table}[h!]
	\caption{Relay signaal die gevalideerd moeten worden}
	\begin{tabular}{llllp{9cm}}
	\toprule
	\textbf{IO} & \textbf{GPIO} & \textbf{Pin} & \textbf{Input/Output} & \textbf{Beschrijving}			\\ \toprule
	Relay		& 3 & 1   	& Output		& Een mechanische apparaat was dicht of opengaat aan de hand of er 3,3 volt of niet op staat.	\\ \bottomrule
	\end{tabular}
	\label{tab:hw_val_relay}
\end{table}
\subsubsection{Testplan}
De relay geeft bij het sluiten en het openen van de mechanische connectie een luid geluid wat op normale afstand van het apparaat te horen moet zijn. Het testplan van de relay wordt dan ook op geluid gedaan. De relay zal dan voor vijf seconden aan gaan, en vervolgens weer voor vijf seconden uitgaan. Hieronder is een overzicht van de relay testplan \ref{tab:hw_val_relay_testplan}.

\begin{table}[h!]
	\caption{Testplan voor de relay}
	\begin{tabular}{lp{14.5cm}}
	\toprule
	\textbf{Naam} 	& \textbf{Verwachte resultaat} \\ \toprule
	Relay			& De microcontroller zet de pin van de relay voor vijf seconden hoog, als dit gebeurt hoor je een klik te horen. Na vijf seconden zal deze pin weer laag gezet worden, dit moet ook gehoord kunnen worden.\\  \bottomrule
	\end{tabular}
	\label{tab:hw_val_relay_testplan}
\end{table}
\newpage

\subsection{Serieel Communicatie}
\subsubsection{Porten}
De Satellite heeft verschillende porten voor seriele communicatie aanbord. De seriele communicatie middelen zullen gebruikt worden voor communicatie met een eindsysteem of sensoren. Sommige sensoren die Sensor Maritime gebruiken sturen de gegevens via seriele communicatie over. Omdat het zoveel functionaliteiten heeft zal dit grondig getest moeten worden. Satellite heeft verschillende varianten van seriele communicatie, ten eerste zijn er twee UART verbindingen, een RS232 verbinding, en uiteindelijk een RS422 communicatie lijn. \newline

\noindent UART staat voor Universal asynchronous receiver-transmitter. UART stuurt asynchroon data over dit betekent dat er geen klok signaal is die data synchroon maakt. UART wordt universeel genoemd omdat het geconfigureerd om verschillende seriele protocollen te ondersteunen. In plaats van een klok signaal wordt er gebruikt gemaakt van een start bit en een stop bit. Data wordt verstuurd via UART via de gebruiker gespecificeerd frequentie, dit wordt ook wel baudrate genoemd, baudrate wordt gespecificeerd als in bits per seconden (bps) \autocite{UART}. \newline

\noindent RS-232 is anders opgebouwd dan UART. RS-232 is een standaard gedefineerd signaal tussen twee apparaten, waar de signaal naam, doel, spanningniveau en pinnen zijn gedefineerd zijn. Een belangrijk verschil tussen RS-232 en UART is dat de spanningniveau hoger en negatief kan zijn. Het voltage bereik van RS-232 gaat -12 volt tot 12 volt \autocite{RS232}. \newline

\noindent RS-422 is weer een andere vorm van seriele communicatie en heeft kenmerken van RS-232. Het verschil tussen RS-232 en RS-422 is dat het voor de communicatie vier lijnen gebruikt (RX-, RX+, TX+, TX-). Met de vier communicatielijnen kan er full duplex gecommuniceerd worden. Full duplex communicatie betekent dat er twee of meer apparaten in beide richting kan communiceren (ontvangen en verzenden op hetzelfde moment) \autocite{FullDuplex}. RS-232 heeft half-duplex dit houdt in dat via RS-232 alleen handeling gedaan kan worden, bijvoorbeeld eerst ontvangen en dan pas verzenden. \autocite{RS422}. \newpage

\noindent In het volgende tabel \ref{tab:hw_val_serieel}wordt een overzicht gegeven van de seriele communicatie die verbonden zijn op de microcontroller. Voor de RS-422 communicatie is alleen twee communicatielijnen aangegeven, omdat dit zo verbonden is op de microcontroller. De Rx+, RX- en TX+, TX- lijnen worden samengevoegd door een chip en dan verbonden naar de microcontroller.
\begin{table}[h!]
	\caption{Seriele communicatie porten die gevalideerd moeten worden}
	\begin{tabular}{llllp{9cm}}
	\toprule
	\textbf{IO} & \textbf{GPIO} & \textbf{Pin} & \textbf{Input/Output} & \textbf{Beschrijving}	\\ \toprule
	UART RX		& 5			& 0    	& Input		& Seriele communicatie ontvangst lijn.			\\
	UART TX		& 5			& 1    	& Output	& Seriele communicatie versturen lijn.			\\
	RS-232 RX	& 3			& 7    	& Input		& Seriele communicatie ontvangst lijn.			\\
	RS-232 TX	& 3			& 8    	& Output	& Seriele communicatie versturen lijn.			\\
	RS-422 RX	& 6			& 3    	& Input		& Seriele communicatie ontvangst lijn.			\\
	RS-422 TX	& 3			& 13   	& Output	& Seriele communicatie versturen lijn.			\\
	UART 2 RX	& 4			& 6    	& Input		& Seriele communicatie ontvangst lijn.			\\
	UART 2 TX	& 4			& 7   	& Output	& Seriele communicatie versturen lijn.			\\ \bottomrule
	\end{tabular}
	\label{tab:hw_val_serieel}
\end{table}

\subsubsection{Testplan}
Om de seriele communicatie te testen wordt er gebruikt gemaakt van GUI. Deze GUI vraagt de om gebruikers input hier in kan de gebruiker een tekst schrijven en dit wordt opgestuurd. De tekst heeft een limiet van maximaal 32 karakters. Er zal hier automatisch een newline karakter toegevoegd worden zodat de microcontroller weet dat dit het einde van het gebruikers bericht is. Het bericht wordt opgestuurd naar de microcontroller en zal de data verwerken. De microcontroller zal dan een antwoord geven als volgt, ten eerste zal de microcontroller vertellen welke variant van communicatie is gebruikt en daarna zal het toevoegen wat de microcontroller ontvangen heeft van de gebruiker. Hieronder is een overzicht \ref{fig:testplanserieel} van hoe het testplan zal werken.

\begin{figure}[h!]
	\caption{Testplan seriele communicatie}
	\label{fig:testplanserieel}
	\includegraphics[width=0.5\linewidth]{voorstudie/testplan/Serieel.jpg}
\end{figure}

